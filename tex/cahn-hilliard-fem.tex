\documentclass[12pt,a4paper]{article}
\usepackage{geometry}
\geometry{left=2.5cm,right=2.5cm,top=2.0cm,bottom=2.5cm}
\usepackage[english]{babel}
\usepackage{amsmath,amsthm}
\usepackage{amsfonts}
\usepackage[longend,ruled,linesnumbered]{algorithm2e}
\usepackage{fancyhdr}
\usepackage{ctex}
\usepackage{array}
\usepackage{listings}
\usepackage{color}
\usepackage{graphicx}
\usepackage{amssymb}
\newtheorem{theorem}{定理}
\newtheorem{lemma}[theorem]{引理}
\newtheorem{corollary}[theorem]{推论}

\begin{document}
	
	\section{Cahn-Hilliard方程的混合有限元形式}
	
	考虑给定的Cahn-Hilliard方程:
	\begin{equation}
		\begin{cases}
			u_t - \Delta\left(-\Delta u + \dfrac{1}{\varepsilon^2} f(u)\right) = 0, & (x, t) \in \Omega \times (0, T], \\
			\left.\partial_n u\right|_{\partial \Omega} = 0, \\
			\left.\partial_n\left(\Delta u - \dfrac{1}{\varepsilon^2} f(u)\right)\right|_{\partial \Omega} = 0, \\
			u|_{t=0} = u_0,
		\end{cases}
	\end{equation}
	其中取 \( f(u) = u^3 - u \),\( \varepsilon \) 为正参数。
	
	在Cahn-Hilliard方程中,自由能 \( E(u) \) 的表达式用于描述系统的能量状态,通常定义为:
	
	\begin{equation}
		E(u) = \int_{\Omega} \left( \frac{\varepsilon^2}{2} |\nabla u|^2 + F(u) \right) dx,
	\end{equation}
	
	其中 \( F(u) = \frac{1}{4}(u^2 - 1)^2 \)
	
	\subsection{引入辅助变量 \( w \)}
	
	为了将高阶的四阶偏微分方程转化为一阶系统,引入辅助变量 \( w \) 定义为:
	\begin{equation}
		w = -\Delta u + \dfrac{1}{\varepsilon^2} f(u)
	\end{equation}
	则原方程可以改写为:
	\begin{equation}
		\begin{cases}
			u_t - \Delta w = 0, & \text{in } \Omega, \\
			w + \Delta u - \dfrac{1}{\varepsilon^2} f(u) = 0, & \text{in } \Omega, \\
			\left.\partial_n u \right|_{\partial \Omega} = 0, & \text{on } \partial \Omega, \\
			\left.\partial_n w \right|_{\partial \Omega} = 0, & \text{on } \partial \Omega.
		\end{cases}
	\end{equation}

	
	\subsection{弱形式}
	
	找 \( u, w \in H^1(\Omega) \),使得 \(\forall q, \psi \in H^1(\Omega) \),满足:
	\begin{align}
		& \int_{\Omega} u_t q \, dx + \int_{\Omega} \nabla w \cdot \nabla q \, dx = 0, & \forall q \in H^1(\Omega), \\
		& \int_{\Omega} \nabla u \cdot \nabla \psi \, dx + \dfrac{1}{\varepsilon^2} \int_{\Omega} f(u) \psi \, dx = \int_{\Omega} w \psi \, dx, & \forall \psi \in H^1(\Omega).
	\end{align}
		
	
	\section{有限元离散}
	
	\subsection{有限元空间}
	
	拉格朗日一次元:
	
	\[
	V_h = \left\{ v \in H^1(\Omega) : v|_{K} \text{ 是线性函数,且在每个单元 } K \text{ 上连续} \right\}
	\]
	
%	\begin{equation}
%		u_h^{n+1} = \sum_{i=1}^N u_i^{n+1} \phi_i, \quad w_h^{n+1} = \sum_{i=1}^N w_i^{n+1} \phi_i
%	\end{equation}
%	
%	这里的 \( u_i^{n+1} \) 和 \( w_i^{n+1} \) 是待求的未知系数,它们表示解 \( u \) 和 \( w \) 在节点 \( i \) 处的近似值。
	
	
%	\subsection{离散化弱形式}
%	
%	找 \( u_h^{n+1}, w_h^{n+1} \in V_h \),使得 \( \forall \phi_j, \phi_k \in V_h \),满足:
%	
%	\begin{align}
%		& \sum_{i=1}^N \dfrac{u_i^{n+1} - u_i^n}{\delta t} \int_{\Omega} \phi_i \phi_j \, dx + \sum_{i=1}^N w_i^{n+1} \int_{\Omega} \nabla \phi_i \cdot \nabla \phi_j \, dx = 0, \quad \forall j=1,2,\ldots,N, \\
%		& \sum_{i=1}^N u_i^{n+1} \int_{\Omega} \nabla \phi_i \cdot \nabla \phi_k \, dx + \dfrac{1}{\varepsilon^2} \sum_{i=1}^N (u_i^n)^3 - u_i^n \int_{\Omega} \phi_i \phi_k \, dx = \sum_{i=1}^N w_i^{n+1} \int_{\Omega} \phi_i \phi_k \, dx, \quad \forall k=1,2,\ldots,N.
%	\end{align}
%	
%	其中,\( u_h^{n+1} = \sum_{i=1}^N u_i^{n+1} \phi_i \) 和 \( w_h^{n+1} = \sum_{i=1}^N w_i^{n+1} \phi_i \) 是在有限元基函数上的展开,满足上式中每个积分形式所规定的离散条件。
	
	
	
	\subsection{离散化弱形式}
	
	找 \( u_h^{n+1} \) 和 \( w_h^{n+1} \in V_h \),使得 \( \forall \phi_j, \phi_k \in V_h \),满足:
	
	\begin{align}
		& \int_{\Omega} \dfrac{u_h^{n+1} - u_h^n}{\delta t} \phi_j \, dx + \int_{\Omega} \nabla w_h^{n+1} \cdot \nabla \phi_j \, dx = 0, \quad \forall j = 1,2,\ldots, N, \\
		& \int_{\Omega} \nabla u_h^{n+1} \cdot \nabla \phi_k \, dx + \dfrac{1}{\varepsilon^2} \int_{\Omega} f(u_h^n) \phi_k \, dx = \int_{\Omega} w_h^{n+1} \phi_k \, dx, \quad \forall k = 1,2,\ldots, N,
	\end{align}
	
	其中 \( f(u_h^n) = (u_h^n)^3 - u_h^n \) 是已知时间步上的非线性项。
	
	接下来,将 \( u_h^{n+1} = \sum_{i=1}^N u_i^{n+1} \phi_i \) 和 \( w_h^{n+1} = \sum_{i=1}^N w_i^{n+1} \phi_i \) 带入上述方程,展开积分项,得到以下离散方程:
	
	\begin{align}
		& \sum_{i=1}^N \dfrac{u_i^{n+1} - u_i^n}{\delta t} \int_{\Omega} \phi_i \phi_j \, dx + \sum_{i=1}^N w_i^{n+1} \int_{\Omega} \nabla \phi_i \cdot \nabla \phi_j \, dx = 0, \quad \forall j=1,2,\ldots,N, \\
		& \sum_{i=1}^N u_i^{n+1} \int_{\Omega} \nabla \phi_i \cdot \nabla \phi_k \, dx + \dfrac{1}{\varepsilon^2} \int_{\Omega} f(u_h^n) \phi_k \, dx = \sum_{i=1}^N w_i^{n+1} \int_{\Omega} \phi_i \phi_k \, dx, \quad \forall k=1,2,\ldots,N.
	\end{align}
	

	
	
	
	\subsection{矩阵表示}
	
	组装下列矩阵:
	
	\begin{itemize}
		\item 质量矩阵 \( M \): \( M_{ij} = \int_{\Omega} \phi_i \phi_j \, dx \)
		\item 刚度矩阵 \( S \):\( S_{ij} = \int_{\Omega} \nabla \phi_i \cdot \nabla \phi_j \, dx \)
		\item 非线性项向量 \( \mathbf{F}(u^n) \):\( F_i(u^n) = \int_{\Omega} f(u_h^n) \phi_i \, dx = \int_{\Omega} \left((u_h^n)^3 - u_h^n\right) \phi_i \, dx \)
	\end{itemize}
	
	将上述定义代入离散方程中,可以得到矩阵形式:
	\begin{equation}
		\begin{cases}
			\frac{1}{\delta t} M (\mathbf{u}^{n+1} - \mathbf{u}^n) + S \mathbf{w}^{n+1} = \mathbf{0}, \\
			S \mathbf{u}^{n+1} + \dfrac{1}{\varepsilon^2} \mathbf{F}(u^n) = M \mathbf{w}^{n+1}。
		\end{cases}
	\end{equation}
	
	\subsection{组合线性系统}
	
	为了同时求解 \( \mathbf{u}^{n+1} \) 和 \( \mathbf{w}^{n+1} \),将上述两个方程组合成一个线性系统:
	\begin{equation}
		\begin{bmatrix}
			\frac{1}{\delta t} M & S \\
			S & -M
		\end{bmatrix}
		\begin{bmatrix}
			\mathbf{u}^{n+1} \\
			\mathbf{w}^{n+1}
		\end{bmatrix}
		=
		\begin{bmatrix}
			\frac{1}{\delta t} M \mathbf{u}^n \\
			- \frac{1}{\varepsilon^2} \mathbf{F}(u^n)
		\end{bmatrix}
	\end{equation}
	
%	这个系统可以表示为:
%	\begin{equation}
%		A
%		\begin{bmatrix}
%			\mathbf{u}^{n+1} \\
%			\mathbf{w}^{n+1}
%		\end{bmatrix}
%		=
%		b
%	\end{equation}
%	其中:
%	\begin{itemize}
%		\item \( A \) 是一个 \( 2N \times 2N \) 的系数矩阵,结构为:
%		\[
%		A = \begin{bmatrix}
%			\frac{1}{\delta t} M & S \\
%			S & -M
%		\end{bmatrix}
%		\]
%		\item \( \mathbf{u}^{n+1} \) 和 \( \mathbf{w}^{n+1} \) 是未知量向量,大小为 \( N \times 1 \)。
%		\item \( b \) 是已知的右端向量,大小为 \( 2N \times 1 \),由当前时间步的解 \( \mathbf{u}^n \) 计算得到:
%		\[
%		b = \begin{bmatrix}
%			\frac{1}{\delta t} M \mathbf{u}^n \\
%			- \frac{1}{\varepsilon^2} \mathbf{F}(u^n)
%		\end{bmatrix}
%		\]
%	\end{itemize}

\section{稳定一阶半隐式格式}

\subsection{弱形式}

找 \( u^{n+1}, w^{n+1} \in H^1(\Omega) \),使得 \( \forall q, \psi \in H^1(\Omega) \),满足:

\begin{align}
	& \frac{1}{\delta t} (u^{n+1} - u^n, q) + (\nabla w^{n+1}, \nabla q) = 0, \quad \forall q \in H^1(\Omega), \\
	& (\nabla u^{n+1}, \nabla \psi) + \frac{S}{\varepsilon^2} (u^{n+1} - u^n, \psi) + \frac{1}{\varepsilon^2} (f(u^n), \psi) = (w^{n+1}, \psi), \quad \forall \psi \in H^1(\Omega).
\end{align}

\subsection{离散化}

选择有限维子空间 \( V_h \subset H^1(\Omega) \),其基函数为 \( \{ \phi_i \}_{i=1}^N \)。我们在每个时间步上寻找 \( u_h^{n+1}, w_h^{n+1} \in V_h \),使得对于所有 \( \phi_j, \phi_k \in V_h \),满足以下离散化后的弱形式方程:

\begin{align}
	& \frac{1}{\delta t} \sum_{i=1}^N (u_i^{n+1} - u_i^n) \int_{\Omega} \phi_i \phi_j \, dx + \sum_{i=1}^N w_i^{n+1} \int_{\Omega} \nabla \phi_i \cdot \nabla \phi_j \, dx = 0, \quad \forall j = 1,2,\ldots, N, \\
	& \sum_{i=1}^N u_i^{n+1} \int_{\Omega} \nabla \phi_i \cdot \nabla \phi_k \, dx + \frac{S}{\varepsilon^2} \sum_{i=1}^N (u_i^{n+1} - u_i^n) \int_{\Omega} \phi_i \phi_k \, dx + \frac{1}{\varepsilon^2} \int_{\Omega} f(u_h^n) \phi_k \, dx = \sum_{i=1}^N w_i^{n+1} \int_{\Omega} \phi_i \phi_k \, dx, \quad \forall k = 1,2,\ldots, N.
\end{align}

\subsection{矩阵形式}

\[
\begin{cases}
	\frac{1}{\delta t} M (\mathbf{u}^{n+1} - \mathbf{u}^n) + S \mathbf{w}^{n+1} = 0, \\
	S \mathbf{u}^{n+1} + \frac{S}{\varepsilon^2} M (\mathbf{u}^{n+1} - \mathbf{u}^n) + \frac{1}{\varepsilon^2} \mathbf{F}(u^n) = M \mathbf{w}^{n+1}.
\end{cases}
\]

\subsection{组合线性系统}

\[
\begin{bmatrix}
	\frac{1}{\delta t} M & S \\
	\frac{S}{\varepsilon^2} M + S & -M
\end{bmatrix}
\begin{bmatrix}
	\mathbf{u}^{n+1} \\
	\mathbf{w}^{n+1}
\end{bmatrix}
=
\begin{bmatrix}
	\frac{1}{\delta t} M \mathbf{u}^n \\
	\frac{S}{\varepsilon^2} M \mathbf{u}^n - \frac{1}{\varepsilon^2} \mathbf{F}(u^n)
\end{bmatrix}.
\]

%在此系统中:
%
%- 左侧的矩阵 \( A \) 是一个 \( 2N \times 2N \) 的系数矩阵,用于描述系统的各项作用。
%- 右侧的向量 \( b \) 是当前时间步的已知项,取决于之前的解 \( \mathbf{u}^n \) 和非线性项 \( \mathbf{F}(u^n) \)。


	
\end{document}


